\section{연습문제 2장 4절: 삼각함수와 거듭제곱급수}

\subsection{2번: $\tan x$의 거듭제곱급수 전개}
다음이 성립함을 알고 있다.\footnote{명시적으로 말하지 않는다면, 이 문서 한정으로 $\sum$만 썼을 때 $n=0$부터 시작한다.}
\begin{align*}
\tan x=\sum \frac{\tan^{(n)}0}{n!}
\end{align*}
이제 거듭제곱급수 전개가 첫 세 항과 같은지 살펴보자. $\tan x$는 홀함수이므로 홀수 번 미분한 값들만 찾으면 된다.
\begin{align*}
\tan^{(1)} 0=\sec^2 0=1
\end{align*}
\begin{align*}
\tan^{(3)} 0=2
\end{align*}
\begin{align*}
\tan^{(5)} 0=16
\end{align*}
따라서 이를 이용하면, $\tan x$의 거듭제곱급수 전개가 다음과 같음을 알 수 있다.
\begin{align*}
\tan x=x+\frac{2x^3}{3!}+\frac{16x^5}{5!}=x+\frac{x^3}{3}+\frac{2x^5}{15}
\end{align*}

\subsubsection{거듭제곱급수 전개의 일반항}
아마 이정도 설명으로는 만족하지 못할 것 같아서 일반항에 대해 좀 더 다루어보자. 사실 $\tan$의 거듭제곱급수의 일반항은 다음과 같다.\footnote{베르누이 수로 나타낼 수도 있지만(그리고 그게 더 `표준적인' 것 같지만) 더 복잡하게 생겼다.}
\begin{align*}
\tan x=\sum\frac{A_{2n+1}x^{2n+1}}{(2n+1)!}
\end{align*}
여기서 $A_n$은 지그재그 수열이라고도 불리는 수열으로\footnote{관심이 있다면 Wolfram Mathworld 같은 곳에서 alternating permutation이라고 검색해 보자.}, 다음과 같이 귀납적으로 정의된다.
\begin{align*}
2A_{n+1}=\sum^n_{k=1}{n\choose k}A_kA_{n-k}
\end{align*}
어떻게 보면 우리 수준에서 일반항을 쉽게 구하지 못하는게 당연한 것 같다. 아, 이 지그재그 수열 $A_n$에는 또 다른 신기한 성질이 있는데, 앞서 보았듯 수열의 홀수 번째 항들은 $\tan$의 거듭제곱급수 전개와 관련이 있고, 짝수 번째 항들은 $\sec$의 거듭제곱급수 전개와 관련이 있다. 식으로 보자면, 다음과 같다.
\begin{align*}
\sec x=\sum\frac{A_{2n}x^{2n}}{(2n)!}
\end{align*}
이러한 특징 때문에 지그재그 수열의 짝수 번째 항들은 시컨트 수열, 홀수 번째 항들은 탄젠트 수열이라고 불리기도 한다.

\subsection{4번 (1)}
다음 극한값을 계산하여야 한다.
\begin{align}\label{prob_4_1}
\lim_{x\to0}\frac{\sin x-x}{x^3}
\end{align}
여기서, $\sin x$의 거듭제곱급수 전개를 대입하면
\begin{align*}
\sin x=x-\frac{x^3}{3!}+\frac{x^5}{5!}-\frac{x^7}{7!}+\cdots
\end{align*}
(\ref{prob_4_1})의 값은 $-1/3!=-1/6$이다.

\subsection{4번 (2)}
다음 극한값을 계산하여야 한다.
\begin{align}\label{prob_4_2}
\lim_{x\to0}\frac{\sin x-x + x^3/6}{x^5}
\end{align}
여기서, $\sin x$의 거듭제곱급수 전개를 대입하면 (\ref{prob_4_2})의 값은 $1/5!=1/120$이다.

\subsection{4번 (3)}
다음 극한값을 계산하여야 한다.
\begin{align}\label{prob_4_3}
\lim_{x\to0}\frac{\cos x-1}{x^2}
\end{align}
여기서 $\cos x$의 거듭제곱급수 전개를 대입하면
\begin{align}
\cos x=1-\frac{x^2}{2!}+\frac{x^4}{4!}-\frac{x^6}{6!}+\cdots
\end{align}
(\ref{prob_4_3})의 값은 $-1/2!=-1/2$이다.

\subsection{4번 (4)}
다음 극한값을 계산하여야 한다.
\begin{align}\label{prob_4_4}
\lim_{x\to0}\frac{\cos x-1+x^2/2}{x^4}
\end{align}
여기서 $\cos x$의 거듭제곱급수 전개를 대입하면 (\ref{prob_4_4})의 값은 $1/4!=1/24$이다.

\subsection{7번: 사인과 코사인의 거듭제곱급수의 성질}
주어진 식을 바꾸면 다음과 같다.
\begin{align*}
\sin x=-\cos\left(x+\frac{\pi}{2}\right)=-\sin\left(x+\pi\right)=\cos\left(x+\frac{3\pi}{2}\right)=\sin(x+2\pi)
\end{align*}
이는 삼각함수의 성질에 의해 성립함이 자명하다.

\section{연습문제 2장 5절: 쌍곡함수}
\subsection{3: 쌍곡함수의 성질 증명}
\subsubsection{$\sinh$의 덧셈정리}
좌변은 다음과 같이 쓸 수 있다.
\begin{align*}
\sinh(x+y)=\frac{e^{x+y}-e^{-x-y}}{2}
\end{align*}
우변을 정의대로 변형하면 쉽게 증명할 수 있다.
\begin{align*}
\sinh x\cosh y+\cosh x\sinh y&=\frac{e^x-e^{-x}}{2}\frac{e^y+e^{-y}}{2}+\frac{e^x+e^{-x}}{2}\frac{e^y-e^{-y}}{2}\\
&=\frac{e^{x+y}-e^{-x+y}+e^{x-y}-e^{-x-y}}{4}+\frac{e^{x+y}+e^{-x+y}-e^{x-y}-e^{-x-y}}{4}\\
&=\frac{e^{x+y}-e^{-x-y}}{2}=\sinh(x+y)
\end{align*}

\subsubsection{$\cosh$의 덧셈정리}
$\sinh$의 덧셈정리와 비슷하게 하면 된다. 하지만 똑같은 계산을 하는것은 재미가 없으니, 좀 다르게 해보자. 다음 식이 성립함을 알고 있다.
\begin{align}\label{cosh_plus_sinh}
\cosh x+\sinh x=e^x
\end{align}
이를 이용하여 좌변을 다음과 같이 변형하자.
\begin{align}
\label{original_lhs}\cosh(x+y)=e^{x+y}-\sinh(x+y)
\end{align}
우변은 다음과 같이 변형할 수 있다.
\begin{align}
\label{original_rhs}\cosh x\cosh y+\sinh x\sinh y&=(e^x-\sinh x)\cosh y+(e^x-\cosh x)\sinh y\\
&=e^x(\cosh y+\sinh y)-(\sinh x\cosh y+\cosh x\sinh y) \nonumber\\
&=e^{x+y}-\sinh(x+y) \nonumber
\end{align}
따라서 (\ref{original_lhs})의 좌변과 (\ref{original_rhs})의 식은 같다.

\subsubsection{$\tanh$의 덧셈정리}
우선 강의에서 $\tanh$에 대한 설명이 없었는데, 삼각함수와 비슷하게 간단하게 정의된다.
\begin{align*}
\tanh x:=\frac{\sinh x}{\cosh x}=\frac{e^x-e^{-x}}{e^x+e^{-x}}=\frac{e^{2x}-1}{e^{2x}+1}
\end{align*}
좌변은 다음과 같이 쓸 수 있다.
\begin{align*}
\tanh(x+y)=\frac{e^{2(x+y)}-1}{e^{2(x+y)}+1}
\end{align*}
우변은 다음과 같이 쓸 수 있다.
\begin{align*}
\frac{\tanh x+\tanh y}{1+\tanh x\tanh y}&=\frac{\frac{e^{2x}-1}{e^{2x}+1}+\frac{e^{2y}-1}{e^{2y}+1}}{1+\frac{e^{2x}-1}{e^{2x}+1}\frac{e^{2y}-1}{e^{2y}+1}}=\frac{(e^{2x}-1)(e^{2y}+1)+(e^{2x}+1)(e^{2y}-1)}{(e^{2x}+1)(e^{2y}+1)+(e^{2x}-1)(e^{2y}-1)}\\
&=\frac{e^{2(x+y)}-1}{e^{2(x+y)}+1}
\end{align*}

\subsubsection{$\cosh$의 반각 공식}
정의를 이용하여 증명할 수도 있지만, 그냥 위에서 증명한 $\cosh$의 덧셈 정리를 이용하자. 다음 식이 성립한다.
\begin{align*}
\cosh(x+x)=\cosh 2x=\cosh x\cosh x+\sinh x\sinh x=\cosh^2 x+\sinh^2 x
\end{align*}
그리고 $\cosh^2 x-\sinh^2 x=1$을 이용하면
\begin{align}\label{cosh_square_plus_sinh_square}
\cosh 2x=\cosh^2 x+(\cosh^2 x-1)=2\cosh^2 x-1
\end{align}
따라서
\begin{align*}
\cosh^2 x=\frac{1}{2}(\cosh 2x+1)
\end{align*}

\subsubsection{$\sinh$의 반각 공식}
$\cosh$의 반각 공식과 비슷하게 하면 되지만, 반복을 피하기 위해 다른 방법을 사용해 보자.\footnote{근데 이것도 반복 아닌가?} (\ref{cosh_square_plus_sinh_square})에 의해 좌변은 다음과 같다.
\begin{align}\label{lhs2}
\sinh^2 x=\cosh^2 x-1
\end{align}
또한, 우변은 다음과 같이 변형할 수 있다.
\begin{align}\label{rhs2}
\frac{1}{2}(\cosh 2x-1)=\frac{1}{2}(\cosh 2x+1)-1
\end{align}
앞서 증명한 $\cosh$의 반각 정리에 의해 (\ref{lhs2})의 좌변과 (\ref{rhs2})의 좌변이 같음을 알 수 있다.

\subsection{6번: 짝함수와 홀함수의 거듭제곱급수 전개}
(i)의 경우 함수 $\mathbf{f}$는 짝함수이고, $\mathbf{f}(x)=\mathbf{f}(-x)$이다. 따라서 $\mathbf{f}$의 거듭제곱급수 전개에서는 짝수항만 나타나야 하므로, 홀수항의 계수인 $f_{2n+1}$은 모두 0이다. (ii)의 경우도 비슷하게 증명할 수 있다. 
