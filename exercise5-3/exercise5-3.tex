\section{연습문제 5장 3절: 벡터의 내적}
\subsection{1번 (2)}
$\mathbf{a}:=(1,1)$, $\mathbf{b}:=(1,\sqrt{3})$이라고 하자.\\[1\baselineskip]
$\mathbf{a}\cdot\mathbf{b}=1+\sqrt{3}$, $|\mathbf{a}|=\sqrt{2}$, $|\mathbf{b}|=2$이므로, 두 벡터 $\mathbf{a}$와 $\mathbf{b}$의 사잇각이 $\phi$일 때, $\mathbf{a}\cdot\mathbf{b}=|\mathbf{a}||\mathbf{b}|\cos\phi$이므로
\[
1+\sqrt{3}=\sqrt{2}\cdot2\cdot\cos\phi \Longleftrightarrow \cos\phi=\frac{\sqrt{2}+\sqrt{6}}{4}
\]

\subsection{1번 (4)}
$\mathbf{a}:=(-1,-2,3)$, $\mathbf{b}:=(4,5,6)$이라고 하자.\\[1\baselineskip]
$\mathbf{a}\cdot\mathbf{b}=4$, $|\mathbf{a}|=\sqrt{14}$, $|\mathbf{b}|=\sqrt{77}$이므로, 두 벡터 $\mathbf{a}$와 $\mathbf{b}$의 사잇각이 $\phi$일 때, $\mathbf{a}\cdot\mathbf{b}=|\mathbf{a}||\mathbf{b}|\cos\phi$이므로
\[
4=\sqrt{14}\cdot\sqrt{77}\cdot\cos\phi \Longleftrightarrow \cos\phi=\frac{2\sqrt{22}}{77}
\]

\subsection{1번 (6)}
$\mathbf{a}:=(0,1,2,3,4)$, $\mathbf{b}:=(0,0,0,0,1)$이라고 하자.\\[1\baselineskip]
$\mathbf{a}\cdot\mathbf{b}=4$, $|\mathbf{a}|=\sqrt{30}$, $|\mathbf{b}|=1$이므로, 두 벡터 $\mathbf{a}$와 $\mathbf{b}$의 사잇각이 $\phi$일 때, $\mathbf{a}\cdot\mathbf{b}=|\mathbf{a}||\mathbf{b}|\cos\phi$이므로
\[
4=\sqrt{30}\cdot1\cdot\cos\phi \Longleftrightarrow \cos\phi=\frac{2\sqrt{30}}{15}
\]

\subsection{3번 (2)}
벡터 $\mathbf{a}$에 대한 벡터 $\mathbf{b}$의 정사영은 다음과 같다.
\[
\frac{\mathbf{a}\cdot\mathbf{b}}{\mathbf{a}\cdot\mathbf{a}}\mathbf{a}=\frac{0}{3}\mathbf{a}=0
\]

\subsection{3번 (4)}
벡터 $\mathbf{a}$에 대한 벡터 $\mathbf{b}$의 정사영은 다음과 같다.
\[
\frac{\mathbf{a}\cdot\mathbf{b}}{\mathbf{a}\cdot\mathbf{a}}\mathbf{a}=\frac{0}{25}\mathbf{a}=0
\]

\subsection{8번: 모든 벡터와 수직인 벡터}
모든 벡터와 수직인 n-벡터를 $\mathbf{v}=(v_1, v_2, \cdots, v_n)$라고 하자. 표준단위벡터 $\mathbf{e}_k$에 대하여, 다음이 성립한다. (단, $1\leq k\leq n$이다.
\[
\mathbf{e}_k\cdot\mathbf{v}=v_k
\]
이때, $\mathbf{v}$는 모든 벡터와 수직이므로 모든 표준단위벡터와도 수직이다. 따라서 $1\leq k\leq n$에 대해, $v_k=0$이고, 벡터 $\mathbf{v}$는 영벡터이다.

\subsection{15번: 코사인 법칙 증명}

다음과 같은 삼각형을 생각하자. 

\begin{center}
\begin{tikzpicture}
\coordinate (A) at (0,0);
\coordinate (B) at (-3,-4);
\coordinate (C) at (4,-3);
\node[label={[label distance=1mm]A}] at (A) {};
\node[label={[anchor=north,label distance=-2mm]B}] at (B) {};
\node[label={[anchor=north,label distance=-2mm]C}] at (C) {};
\draw[thick,->] (A) -- (B) node[midway,above,label={{[label distance=2mm]$\mathbf{a}$}}] {};
\draw[thick,->] (A) -- (C) node[midway,above,label={{[label distance=2mm]$\mathbf{b}$}}] {};
\draw[thick,->] (B) -- (C) node[midway,below,label={{[label distance=-6mm]$\mathbf{c}$}}] {};
\pic [draw, "$\theta$", angle eccentricity=1.5] {angle = B--A--C};
\end{tikzpicture}
\end{center}

그림에서 알 수 있듯, $\mathbf{a}=\overrightarrow{AB}$, $\mathbf{b}=\overrightarrow{AC}$, $\mathbf{c}=\overrightarrow{BC}$이다. 이때 $\mathbf{c}=\mathbf{b}-\mathbf{a}$이고, 따라서 다음이 성립한다.
\[
|\mathbf{c}|^2=|\mathbf{b}-\mathbf{a}|^2=(\mathbf{b}-\mathbf{a})\cdot(\mathbf{b}-\mathbf{a})=|\mathbf{b}|^2-2\mathbf{b}\cdot\mathbf{a}+|\mathbf{a}|^2
\]

여기서 $|\mathbf{a}|=a$, $|\mathbf{b}|=b$, $|\mathbf{c}|=c$일 때, 위 식은 다음과 같다.
\[
c^2=a^2+b^2-2ab\cos\theta
\]
