\section{연습문제 7장 1절: 역행렬}
\subsection{1번: 정사각행렬 교환하기}
수업에 나왔던 것처럼 반례가 존재한다.
\begin{align*}
A=\begin{pmatrix}
1 & 1\\
0 & 0
\end{pmatrix},
B=\begin{pmatrix}
1 & 0\\
-1 & 0
\end{pmatrix}
\end{align*}
\begin{align*}
AB=O, BA=\begin{pmatrix}1 & 1\\-1 & -1\end{pmatrix}\neq O
\end{align*}

\subsection{2번: 행렬 곱하기}
그냥 곱해보자.
\begin{align*}
AB=\begin{pmatrix}1 & 2 & 0\\2 & 3 & 0\end{pmatrix}\begin{pmatrix}-3 & 2\\2 & -1\\0 & 0\end{pmatrix}=\begin{pmatrix}1 & 0\\0 & 1\end{pmatrix}=I
\end{align*}
묻고 떠블로 곱해보자.
\begin{align*}
BA=\begin{pmatrix}-3 & 2\\2 & -1\\0 & 0\end{pmatrix}\begin{pmatrix}1 & 2 & 0\\2 & 3 & 0\end{pmatrix}=\begin{pmatrix}1 & 0 & 0\\0 & 1 & 0\\0 & 0 & 0\end{pmatrix}\neq I
\end{align*}

\subsection{3번: 가역행렬의 성질}
\subsubsection{$(A^{-1})^{-1}=A$ 증명}
$A^{-1}$의 역행렬을 $B$라고 가정하자. 이때 다음이 성립한다.
\begin{align*}
A^{-1}B=BA^{-1}=I
\end{align*}
이를 만족하는 $B$는 $A$만 존재한다. 따라서 $(A^{-1})^{-1}=A$이다.

\subsubsection{$(A^{-1})^k=(A^k)^{-1}$ 증명}
역행렬의 정의에 의해, 다음이 성립함을 알고 있다.
\begin{align}\label{pow_invA_k}
(A^{-1})^kA^k=(AA^{-1})^k=I^k=I, A^k(A^{-1})^k=(A^{-1}A)^k=I^k=I
\end{align}
한편, 다음 또한 성립한다.
\begin{align}\label{inv_pow_A_k}
(A^k)^{-1}A^k=I, A^k(A^k)^{-1}=I
\end{align}
(\ref{pow_invA_k}), (\ref{inv_pow_A_k})에 의해 $(A^{-1})^k$과 $(A^k)^{-1}$의 역행렬은 모두 $A^k$로 같음을 알 수 있고, 역행렬이 같으므로 두 행렬도 같다.

\subsubsection{$(A^t)^{-1}=(A^{-1})^k$ 증명}
방금 전과 비슷한 논리를 활용해 보자. 다음이 성립함을 알고 있다.
\begin{align}\label{transpose_invA}
(A^t)^{-1}A^t=I, A^t(A^t)^{-1}=I
\end{align}
또한, 다음 또한 전치행렬의 성질 때문에 성립한다.
\begin{align}\label{inv_transposed_A}
A^t(A^{-1})^t=(A^{-1}A)^t=I^t=I, (A^{-1})^tA^t=(AA^{-1})^t=I^t=I
\end{align}
(\ref{transpose_invA}), (\ref{inv_transposed_A})에 의해 $(A^t)^{-1}$와 $(A^{-1})^t$의 역행렬은 모두 $A^t$로 같음을 알 수 있고, 역행렬이 같으므로 두 행렬도 같다.

\subsubsection{$(AB)^{-1}=B^{-1}A^{-1}$ 증명}
행렬 $A$, $B$로 얻은 선형사상을 각각 $T_A$, $T_B$라 하자. 이때 $AB$로 얻은 사상은 $T_A\circ T_B$와 같음을 알고 있으므로, $(AB)^{-1}$로 얻은 사상은 $(T_A\circ T_B)^{-1}$과 같다. 한편, $A^{-1}$, $B^{-1}$로 얻은 사상은 각각 ${T_A}^{-1}$, ${T_B}^{-1}$와 같다. 이때 $B^{-1}A^{-1}$로 얻은 사상은 ${T_A}^{-1}\circ {T_B}^{-1}$이고, 이는 다음과 같은 법칙에 의해 앞에서 구한 $(T_A\circ T_B)^{-1}$과 같음을 알 수 있다.

\begin{displayquote}
신을 때에는 양말 신고 신 신고, 벗을 때에는 신 벗고 양말 벗고 (김홍종, 미적분학 1+, p. 86)
\end{displayquote}

\section{연습문제 7장 3절: 치환}
\subsection{1번: 행렬식 계산}
\subsubsection{행렬식의 성질 이용하기}
지금까지 배운 행렬식의 성질을 이용하자.
\begin{align*}
\det\begin{pmatrix}
1 & 2 & 3\\
4 & 5 & 6\\
7 & 8 & 9
\end{pmatrix}=\det\begin{pmatrix}
1 & 1 & 3\\
4 & 1 & 6\\
7 & 1 & 9
\end{pmatrix}=\det\begin{pmatrix}
1 & 1 & 2\\
4 & 1 & 2\\
7 & 1 & 2
\end{pmatrix}=2\det\begin{pmatrix}
1 & 1 & 1\\
4 & 1 & 1\\
7 & 1 & 1
\end{pmatrix}=0
\end{align*}
\begin{align}
\det\begin{pmatrix}
1 & 9 & 99\\
0 & 2 & 88\\
0 & 0 & 3
\end{pmatrix}&=\det\begin{pmatrix}
1 & 2 & 99\\
0 & 2 & 88\\
0 & 0 & 3
\end{pmatrix}=2\det\begin{pmatrix}
1 & 1 & 99\\
0 & 1 & 88\\
0 & 0 & 3
\end{pmatrix}=2\det\begin{pmatrix}
1 & 1 & 3\\
0 & 1 & 3\\
0 & 0 & 3
\end{pmatrix} \nonumber\\
\label{triangular_matrix}&=3\times2\det\begin{pmatrix}
1 & 1 & 1\\
0 & 1 & 1\\
0 & 0 & 1
\end{pmatrix}=3\times2\times1\times1\times1=6
\end{align}
(\ref{triangular_matrix})의 결과는 삼각행렬의 행렬식을 이용하면 얻을 수 있다.

\begin{align*}
\det\begin{pmatrix}
1 & 1 & 1\\
1 & 2 & 3\\
1 & 4 & 9
\end{pmatrix}&=\det\begin{pmatrix}
1 & 0 & 0\\
1 & 1 & 2\\
1 & 3 & 8
\end{pmatrix}=-\det\begin{pmatrix}
0 & 0 & 1\\
2 & 1 & 1\\
8 & 3 & 1
\end{pmatrix}=\det\begin{pmatrix}
8 & 3 & 1\\
2 & 1 & 1\\
0 & 0 & 1
\end{pmatrix}\\
&=\det\begin{pmatrix}
8 & 3 & 1\\
0 & 1/4 & 3/4\\
0 & 0 & 1
\end{pmatrix}=8\times\frac{1}{4}\times1=2
\end{align*}

\subsubsection{라플라스 전개}
이충훈 선생님께서 기하 시간\footnote{당연히 2020년 1학기 기준이다.}에 외적을 구할 때에도 잠깐 보여 주셨지만, 행렬식을 쉽게 계산하는 방법이 있다. 행렬식의 성질을 어떻게 사용할지 생각이 안 날 때에는 요긴하게 사용할 수 있을 것이다. 다음 행렬의 행렬식을 한 번 라플라스 전개로 계산해 보자.\footnote{이게 왜 가능한지는 \href{https://en.wikipedia.org/wiki/Laplace_expansion\#Proof}{증명}을 참고하자. 라플라스 전개를 알고 있으면 벡터곱을 배울 때 책에 나오는 공식 일부를 외울 필요 없이 계산이 쉬워진다.}
\begin{align*}
\begin{pmatrix}
1 & 2 & 3\\
4 & 5 & 6\\
7 & 8 & 9
\end{pmatrix}
\end{align*}
\paragraph{행/열 정하기} 우선 주어진 행렬에서 열이나 행을 하나 정해야 한다. 여기에서는 그냥 첫 번째 열을 고르자. 어떤 행이나 열을 고르든 계산의 결과는 같다.
\paragraph{행렬 분해하기} 앞서 고른 행이나 열의 각각의 항에 대해, 각각의 항들을 지나는 열과 행을 지우고, 남은 항들로 새로운 행렬을 만든다. $n$차 정사각행렬이라면, $n-1$차 정사각행렬이 $n$개 만들어질 것이다. 여기서 예로 든 정사각행렬로 보면, 파란색으로 되어 있는 항을 기준으로 분해할 때를 보면 다음과 같다. 빨간색으로 표시된 항들은 분해 과정에서 없어지는 항들이다.
\begin{align*}
\begin{pmatrix}
\color{blue}1 & \color{red}2 & \color{red}3\\
\color{red}4 & 5 & 6\\
\color{red}7 & 8 & 9
\end{pmatrix}\longrightarrow\begin{pmatrix}
5 & 6\\
8 & 9
\end{pmatrix},
\begin{pmatrix}
\color{red}1 & 2 & 3\\
\color{blue}4 & \color{red}5 & \color{red}6\\
\color{red}7 & 8 & 9
\end{pmatrix}\longrightarrow\begin{pmatrix}
2 & 3\\
8 & 9
\end{pmatrix},
\begin{pmatrix}
\color{red}1 & 2 & 3\\
\color{red}4 & 5 & 6\\
\color{blue}7 & \color{red}8 & \color{red}9
\end{pmatrix}\longrightarrow\begin{pmatrix}
2 & 3\\
5 & 6
\end{pmatrix}
\end{align*}
\paragraph{행렬식 나눠서 계산하기} 분해된 각각의 행렬의 행렬식을 구하고, 행렬을 분해할 때 기준으로 잡은 항을 곱해준다.
\begin{align*}
1\det\begin{pmatrix}
5 & 6\\
8 & 9
\end{pmatrix}=-3,
4\det\begin{pmatrix}
2 & 3\\
8 & 9
\end{pmatrix}=-24,
7\det\begin{pmatrix}
2 & 3\\
5 & 6
\end{pmatrix}=-21
\end{align*}
\paragraph{부호 붙여서 결과값 내기} 이제 각각의 결과값에 부호를 붙여서 더하면 된다. 기준으로 잡은 항이 $(i, j)$항일 때, 앞서 구한 행렬식의 값들에 $(-1)^{i+j}$를 곱해준다. 이제 행렬식의 값을 구할 수 있다.
\begin{align*}
\det\begin{pmatrix}
1 & 2 & 3\\
4 & 5 & 6\\
7 & 8 & 9
\end{pmatrix}&=1\det\begin{pmatrix}
5 & 6\\
8 & 9
\end{pmatrix}-4\det\begin{pmatrix}
2 & 3\\
8 & 9
\end{pmatrix}+7\det\begin{pmatrix}
2 & 3\\
5 & 6
\end{pmatrix}\\
&=-3+24-21=0
\end{align*}

\subsection{2번: 나란히꼴의 넓이}
우선 $\mathbf{x}=(x_1, x_2)$, $\mathbf{y}=(y_1, y_2)$라고 하자. 이때 나란히꼴의 넓이는 삼각형의 넓이 공식 $\dfrac{1}{2}ab\sin C$를 이용하면 넓이가 다음과 같음을 알 수 있다. ($\varphi$를 두 벡터의 사잇각이라고 두자.)
\begin{align*}
|\mathbf{x}||\mathbf{y}|\sin\varphi
\end{align*}
두 벡터의 내적을 이용하면, $\cos\varphi$를 계산할 수 있다.
\begin{align*}
\cos\varphi=\frac{\mathbf{x}\cdot\mathbf{y}}{|\mathbf{x}||\mathbf{y}|}=\frac{x_1y_1+x_2y_2}{\sqrt{{x_1}^2+{x_2}^2}\sqrt{{y_1}^2+{y_2}^2}}
\end{align*}
이제 $\sin\varphi$를 계산하면 다음과 같다.
\begin{align*}
\sin\varphi=\sqrt{1-\cos^2\varphi}=\sqrt{1-\frac{(x_1y_1+x_2y_2)^2}{({x_1}^2+{x_2}^2)({y_1}^2+{y_2}^2)}}=\frac{|x_1y_2-x_2y_1|}{\sqrt{{x_1}^2+{x_2}^2}\sqrt{{y_1}^2+{y_2}^2}}
\end{align*}
따라서 넓이는 다음과 같다.
\begin{align*}
|\mathbf{x}||\mathbf{y}|\sin\varphi=\sqrt{{x_1}^2+{x_2}^2}\sqrt{{y_1}^2+{y_2}^2}\cdot \frac{|x_1y_2-x_2y_1|}{\sqrt{{x_1}^2+{x_2}^2}\sqrt{{y_1}^2+{y_2}^2}}=|x_1y_2-x_2y_1|
\end{align*}
이때, $|x_1y_2-x_2y_1|$을 다르게 나타내면
\begin{align*}
|x_1y_2-x_2y_1|=\left|\det\begin{pmatrix}
x_1 & y_1\\
x_2 & y_2
\end{pmatrix}\right|=|\det(\mathbf{x}, \mathbf{y})|
\end{align*}

\subsection{3번: '신발끈 공식' 증명}
1학년 도형의 방정식 단원에서 `꼼수'로 자주 사용했던 `신발끈 공식'을 증명해 보자. 점과 직선 사이 거리 공식을 통해 유도하는 것이 정석이지만, 여기에서는 앞서 증명한 나란히꼴의 넓이 공식을 이용하자. 우선 점 $(x_1, y_1)$과 $(x_2, y_2)$, 원점으로 이루어진 삼각형의 넓이는 나란히꼴의 넓이의 절반이므로, 다음과 같다.
\begin{align*}
\frac{1}{2}|x_1y_2-x_2y_1|
\end{align*}
그렇다면 여기에서 원점이 $(x_3, y_3)$으로 바뀐다면 어떨까? 직접 구하기는 귀찮겠지만, 평행이동을 하면 이 삼각형의 넓이는 $(x_1-x_3, y_1-y_3)$과 $(x_2-x_3, y_2-y_3)$, 원점으로 이루어진 삼각형의 넓이와 같다.
\begin{align}
\frac{1}{2}|(x_1-x_3)(y_2-y_3)-(x_2-x_3)(y_1-y_3)|&=\frac{1}{2}|(x_2y_3-x_3y_2)-(x_1y_3-y_1x_3)+(x_1y_2-y_1x_2)| \nonumber\\
\label{poly_to_det}&=\frac{1}{2}\left|\det\begin{pmatrix}
x_1 & y_1 & 1\\
x_2 & y_2 & 1\\
x_3 & y_3 & 1
\end{pmatrix}\right|
\end{align}
(\ref{poly_to_det})의 결과는 앞서 설명한 라플라스 전개를 생각하면 쉽게 이해 가능하다. 
