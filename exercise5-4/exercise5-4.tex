\section{연습문제 5장 4절: 평면과 직선의 방정식, 무게중심}
\subsection{1번: 직선의 수직조건}
두 직선 $y=mx+b$와 $y=m'x+b'$은 각각 $(m, -1)\cdot\{(x,y)-(0,-b)\}=0$, $(m', -1)\cdot\{(x,y)-(0,-b')\}=0$이라는 초평면으로 해석할 수 있다. 두 초평면의 방향벡터가 수직하면 두 평면은 수직하므로\footnote{연습문제 7번에서 이를 다룬다.}, $(m,-1)\cdot(m',-1)=0$, 따라서 $mm'=-1$은 두 직선이 수직한 것의 필요충분조건.

\subsection{2번: 수직 판별}
앞에서 증명한 사실을 이용하면 된다.

\subsection{3번: 평면에 대한 점의 사영}
우선 주어진 평면 $ax+by+cz=d$를 $(a,b,c)\cdot\{(x,y,z)-(d/a,0,0)\}=0$으로 나타낼 수 있고, 여기에서 $A=(d/a,0,0)$이라고 두고 $(a,b,c)\cdot\{(x,y,z)-A\}=0$라 하자.
\begin{center}
\definecolor{zzttff}{rgb}{0.6,0.2,1}
\definecolor{ududff}{rgb}{0.30196078431372547,0.30196078431372547,1}
\definecolor{qqwuqq}{rgb}{0,0.39215686274509803,0}
\definecolor{xdxdff}{rgb}{0.49019607843137253,0.49019607843137253,1}
\begin{tikzpicture}[line cap=round,line join=round,>=triangle 45,x=1cm,y=1cm]
\clip(-3.1021446747716483,-3.670973143225158) rectangle (5.5986987379819935,3.0172509346591934);
\draw[line width=1pt,color=qqwuqq,fill=qqwuqq,fill opacity=0.10000000149011612] (-0.08337722045083042,-0.16675444090166083) -- (0.08337722045083043,-0.25013166135249126) -- (0.16675444090166083,-0.08337722045083043) -- (0,0) -- cycle; 
\draw [line width=1pt,domain=-3.1021446747716483:5.5986987379819935] plot(\x,{(-0--2*\x)/1});
\draw [->,line width=1pt] (0,0) -- (2.6,-1.3);
\draw [->,line width=1pt] (4,0) -- (0.8,1.6);
\draw [->,line width=1pt] (4,0) -- (-1.5,-3);
\begin{scriptsize}
\draw[color=black] (-0.5,-3.525959086345931) node {$ax+by+cz=d$};
\draw [fill=xdxdff] (-1.5,-3) circle (2.5pt);
\draw[color=xdxdff] (-1.8,-2.8140718980297255) node {$A$};
\draw[color=black] (1.3449530695246574,-0.5114244494019986) node {$\mathbf{n}$};
\draw [fill=ududff] (4,0) circle (2.5pt);
\draw[color=ududff] (4.2,0.3) node {$O$};
\draw [fill=zzttff] (0.8,1.6) circle (2pt);
\draw[color=zzttff] (0.5,1.773645537785822) node {$P$};
\draw[color=black] (2.8,1.2) node {$p_\mathbf{n}(\overrightarrow{OA})$};
\draw[color=black] (1.5,-1.7) node {$\overrightarrow{OA}$};
\end{scriptsize}
\end{tikzpicture}
\end{center}
그림에서 알 수 있듯, $\overline{OP}$는 벡터 $\overrightarrow{OA}$를 $\mathbf{n}$에 정사영한 것이다.

\subsection{9번: 입체사영 증명}
원통 좌표계를 이용하여 입체사영의 관계식을 유도해 보자.\footnote{원통좌표계를 사용한 것은 구에서 오는 대칭성을 최대한 이용하기 위해서이다.} 입체 사영은 구를 평면에 사영하는 대응(함수)으로 생각할 수 있다. 다음 그림을 보자.
\begin{center}
\definecolor{uuuuuu}{rgb}{0.26666666666666666,0.26666666666666666,0.26666666666666666}
\definecolor{xdxdff}{rgb}{0.49019607843137253,0.49019607843137253,1}
\begin{tikzpicture}[line cap=round,line join=round,>=triangle 45,x=1cm,y=1cm]
\begin{axis}[
x=1cm,y=1cm,
axis lines=middle,
xmin=-5,
xmax=3,
ymin=-3,
ymax=3,
ticks=none]
\clip(-9.9,-8.02) rectangle (9.9,8.02);
\draw [line width=1pt] (0,0) circle (2cm);
\draw [line width=1pt] (0,2)-- (-4,0);
\draw [line width=1pt] (-1.6,1.2)-- (-1.6,0);
\begin{scriptsize}
\draw [fill=xdxdff] (-4,0) circle (2.5pt);
\draw[color=xdxdff] (-4,0.43) node {$Q$};
\draw [fill=uuuuuu] (-1.6,1.2) circle (2pt);
\draw[color=uuuuuu] (-1.6,1.6) node {$P$};
\draw [fill=uuuuuu] (-1.6,0) circle (2pt);
\draw[color=uuuuuu] (-1.6,-0.4) node {$H$};
\draw [fill=uuuuuu] (0,2) circle (2pt);
\draw[color=uuuuuu] (0.2,2.3) node {$A$};
\draw[color=uuuuuu] (0.2,0.2) node {$O$};
\end{scriptsize}
\end{axis}
\end{tikzpicture}
\end{center}
그림을 설명하자면, 반지름의 길이가 1이고, 중심이 원점인 구가 있을 때, 구 위의 어느 점 $P$와 중심을 포함하는 단면으로 자른 모습이다. 입체사영을 하면, 구 위의 점 $P$는 $xy$평면 위의 점 $Q$로 대응된다. 앞서 설명한 원통 좌표계를 이용하여 $P$와 $Q$, $A$를 각각 $(r, \theta, z)$, $(R, \Theta, 0)$, $(0,0,1)$으로 나타낼 수 있다. 점 $P$의 $xy$평면에 대한 정사영을 $H$라 하자. 이때 $\triangle{AQO}$와 $\triangle{PQH}$는 닮음이고, 닮음비는 $1:z$이다. 이를 이용하면 다음과 같은 식이 성립한다.
\begin{align*}
\Seg{QO}:\Seg{QH}=1:z\Longleftrightarrow\Seg{QO}:\Seg{HO}=R:r=1:(1-z)
\end{align*}
$A$, $Q$, $O$는 같은 평면에 있으므로 $\Theta=\theta$이고, 따라서 $Q$의 좌표는
\begin{align*}
\left(\frac{r}{1-z}, \theta, 0\right)
\end{align*}
으로 쓸 수 있다. 직교 좌표계로 $Q$를 나타내면 다음과 같다.
\begin{align*}
\left(\frac{r\cos\theta}{1-z}, \frac{r\sin\theta}{1-z}, 0\right)
\end{align*}
그런데 $P$를 직교 좌표계로 나타내면 다음과 같으므로
\begin{align*}
(x, y, z)=\left(r\cos\theta, r\sin\theta, z\right)
\end{align*}
$Q$의 좌표는 다음과 같이 쓸 수 있다.
\begin{align*}
\left(\frac{x}{1-z}, \frac{y}{1-z}, 0\right)
\end{align*}
