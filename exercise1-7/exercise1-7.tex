\section{연습문제 1장 7절: 교대급수와 절대수렴급수}

\subsection{1번 (4)}
다음과 같이 놓을 때,

\[
a_n:=(-1)^n\sin \frac{1}{n}
\]

$\lim_{n\to\infty}a_n=0$이므로 급수 $\sum a_n$은 수렴한다. 절대수렴하는지 알아보기 위해, 다음과 같이 정의하자.

\[
b_n:=|a_n|=\sin \frac{1}{n}
\]

오랜만에 비교판정법을 쓰면 된다. $\sin x>2x/\pi$이므로 $b_n=\sin(1/n)>2/n\pi$이다. 조화급수는 발산하므로 급수 $\sum b_n$도 발산. 절대수렴하지는 않는다.

\subsection{1번 (5)}
우선, $a_n:=(-1)^{n+1}/n^{3/2}$일 때, $\lim_{n\to\infty}a_n=0$이므로 급수 $\sum a_n$은 수렴. $b_n:=|a_n|=1/n^{3/2}$일 때, 비율판정법을 이용하여 급수가 수렴하는지 알아보자.

\[
\frac{b_{n+1}}{b_n}=\frac{n^{3/2}}{(n+1)^{3/2}}=\left(\frac{n}{n+1}\right)^{3/2}
\]

라고 했지만 $b_{n+1}/b_n$이 1에 수렴하기 때문에, 그렇게 할 수 없다. 이를 증명하는 방법은 여러 가지가 있는데, 가장 재미없는 것부터 살펴보자.

\subsubsection{적분판정법 이용하기 (노잼편)}

배우지도 않은 것을 쓰는 것이 좋은 방법은 아니지만, 사실 대부분의 미적분 관련 책에서는 적분판정법으로 증명하므로 넣었다.\footnote{이거 쓴 사람이 적분판정법 써서 풀고 제출한건 안비밀}적분하는 것이 가장 쉬운 방법이기도 하다. 적분판정법은 책 1장 6절에 있으므로 그곳에 있는 설명을 일단 읽어보자.

\[
\int^\infty_1 x^{-3/2}dx=2
\]

적분값이 존재하므로, 급수 $\sum b_n$은 수렴. $\sum a_n$은 절대수렴.

\subsubsection{신박한 방법}

이 문제를 푼 이후, 다른 사람들의 풀이 방법이 궁금해서 여기저기 검색을 해 보았는데, 여기서는 검색한 것들 중 가장 신기했던 방법을 다루어 보자.\footnote{\href{https://math.stackexchange.com/questions/29450/self-contained-proof-that-sum-limits-n-1-infty-frac1np-converges-for}{Math Stack Exchange에서 이 문제와 관련된 질문을 볼 수 있다.}}

\[
S_n:=\sum^n_{k=1}\frac{1}{k^{3/2}}
\]

부분합을 이렇게 정의할 때, 다음이 성립한다.

\begin{align*}
S_{2n+1}=\sum^{2n+1}_{k=1}\frac{1}{k^{3/2}}=&1+\sum^n_{k=1}\left\{\frac{1}{(2k)^{3/2}}+\frac{1}{(2k+1)^{3/2}}\right\}\\
<&1+\sum^n_{k=1}\frac{2}{(2k)^{3/2}}=1+2^{1-(3/2)}S_n\\
<&1+2^{1-(3/2)}S_{2n+1}
\end{align*}

$S_{2n+1}$에 대하여 정리하면 다음과 같다.

\[
S_{2n+1}<\frac{1}{1-2^{1-(3/2)}}
\]

수열 $(S_n)$은 단조 증가하고, 위로 유계이므로 수렴한다.

\subsection{1번 (6)}
멀리 갈 것도 없이, $\lim_{n\to\infty}(n/10)^n$과 $\lim_{n\to\infty}(-1)^{n+1}(n/10)^n$ 모두 0이 아니므로 발산한다.

\subsection{1번 (9)}
다음과 같이 놓을 때,

\[
a_n:=(-1)^n\log \left(1+\frac{1}{n}\right)=(-1)^n\log \frac{n+1}{n}
\]

$\lim_{n\to\infty}a_n=0$이므로, 급수 $\sum a_n$은 수렴. 절대수렴하는지를 알아보기 위해, $b_n=|a_n|$으로 놓자. 이때, 로그의 성질에 의해

\[
\sum^n_{k=1} b_k=\log \frac{2}{1}+\log \frac{3}{2}+\log \frac{4}{3}+\cdots=\log n
\]

따라서 $\lim_{n\to\infty}\sum^n_{k=1}b_n=\sum b_n=\infty$

\subsection{1번 (10)}
우선, $a_n:=(-1)^n/\log n$이라 해 보자. 이때, $\lim_{n\to\infty}a_n=0$이므로, $\sum a_n$은 수렴. 또, $b_n=|a_n|$이라고 놓을 때, $b_n=1/\log n > 1/n$이므로, $\sum b_n$은 발산, $\sum a_n$은 절대수렴하지 않음.
